\documentclass[11pt,twocolumn]{article}

\usepackage[utf8]{inputenc}
\usepackage[spanish]{babel}
\usepackage{url}

\title{Evolución de Wikimedia Commons}
\author{emijrp (emijrp@gmail.com)}
\date{\today}

\begin{document}
\maketitle

%Este paper surge de la idea de este post: http://emijrp.blogspot.com/2007/10/ms-imgenes-y-mayor-resolucin.html
%Muestra cómo el material que se sube a Commons cada vez ocupa más espacio,
%el tamaño medio de las imágenes se incrementa con el paso del tiempo,
%y las imágenes tienen mayor resolución (debido a que la gente tiene mejores cámaras cada vez).
%La idea es formalizar todo esto (el post está escrito un poco a la charamanduska)
%mejorar las gráficas (usar R o similar) y ofrecer enlaces al código (incluso mostrar algún fragmento?).

%TODO:
%jugar con los metadatos de las imágenes

\section{\uppercase{Introducción}}

%Qué es Wikimedia Commons, situarlo en el contexto de Wikipedia, la web 2.0, los hostings de imágenes públicos (Flickr),
%recalcar la libertad de sus contenidos

Con la llegada de la segunda generación de la web, los usuarios se convirtieron en productores de información. Esta información se encontraba en formato texto, imagen o video. Estos últimos se hicieron cada vez más famosos debido a la proliferación de las cámaras digitales, cuyos precios se abarataron hasta el punto de que con un reducido presupuesto se podía adquirir una cámara de características semiprofesionales.

Para cubrir la necesidad de publicación de estos formatos, surgieron sitios web especializados en albergar este tipo de contenidos. En el caso de las imágenes (aunque no restringido a ellas) aparecieron Flickr o ... En el lado de los videos, YouTube es el más conocido.

Wikimedia Commons\footnote{\url{http://commons.wikimedia.org}} es uno de estos repositorios, aunque con una particularidad, sólo alberga contenido con licencias libres (recalcando la condición de permitir el uso comercial) y que estén dentro del \textit{scope} del proyecto, esto es, que tenga un uso potencial en los artículos de Wikipedia o sus proyectos hermanos.

\section{\uppercase{Evolución}}

\section{\uppercase{Conclusiones y trabajo futuro}}

\section{\uppercase{Referencias}}

\end{document}
