\documentclass[twoside]{article}


% ------
% Fonts and typesetting settings
\usepackage[sc]{mathpazo}
\usepackage[T1]{fontenc}
\linespread{1.05} % Palatino needs more space between lines
\usepackage{microtype}


% ------
% Page layout
\usepackage[hmarginratio=1:1,top=32mm,columnsep=20pt]{geometry}
\usepackage[font=it]{caption}
\usepackage{paralist}
\usepackage{multicol}

% ------
% Lettrines
\usepackage{lettrine}


% ------
% Abstract
\usepackage{abstract}
	\renewcommand{\abstractnamefont}{\normalfont\bfseries}
	\renewcommand{\abstracttextfont}{\normalfont\small\itshape}


% ------
% Titling (section/subsection)
\usepackage{titlesec}
\titleformat{\section}[block]{\large\scshape\centering{\Roman{section}.}}{}{1em}{}


% ------
% Header/footer
\usepackage{fancyhdr}
	\pagestyle{fancy}
	\fancyhead{}
	\fancyfoot{}
	\fancyhead[C]{Journal paper template $\bullet$ October 2012 $\bullet$ Vol. I, No. 1}
	\fancyfoot[RO,LE]{\thepage}


% ------
% Clickable URLs (optional)
\usepackage{hyperref}

% ------
% Maketitle metadata
\title{\vspace{-15mm}%
	\fontsize{24pt}{10pt}\selectfont
	\textbf{WikiPapers: a collaborative compilation of wiki research literature... in a wiki!}\thanks{Consulta y edita una versi\'{o}n actualizada de esta publicaci\'{o}n: \href{http://wikipapers.referata.com/wiki/ROI:1}{http://wikipapers.referata.com/wiki/ROI:1}}
	}	
\author{%
	\large
	\textsc{Emilio J. Rodr\'{i}guez-Posada} \\ 
	\normalsize	Private 
	\normalsize	\href{mailto:emijrp@gmail.com}{emijrp@gmail.com}
	\vspace{-5mm}
	}
\date{}



%%%%%%%%%%%%%%%%%%%%%%%%
\begin{document}

\maketitle
\thispagestyle{fancy}

\begin{abstract}
\noindent El inter\'{e}s de los investigadores por los wikis, en especial Wikipedia, ha ido creciendo en los \'{u}ltimos a\~{n}os. La primera edici\'{o}n de WikiSym, un simposio sobre wikis, se celebr\'{o} en 2005 y desde entonces han aparecido congresos como CLEF/PAN Lab, workshops como WikiAI, SemWiki y MathWikis, conferencias como Wikimania, WikiCon, SMWCon, Wiki Conference India, Wikipedia Academy y Wikipedia CPOV Conference y competiciones como WikiViz. El estudio de los wikis es un campo emergente y prol\'{i}fico. Por ello, ha habido varios intentos de recopilar toda la literatura cient\'{i}fica sobre wikis, aunque con escaso \'{e}xito. Unas veces el enfoque o la herramienta utilizada eran limitados, otras veces el proyecto era abandonado y al poco tiempo los metadatos bibliogr\'{a}ficos se perd\'{i}an. En este art\'{i}culo presentamos WikiPapers, un proyecto colaborativo para recopilar toda la literatura cient\'{i}fica sobre wikis. Se hace uso de MediaWiki y su extensi\'{o}n sem\'{a}ntica, ambos conocidos por los investigadores de este \'{a}rea. Hasta octubre de 2012 se han recopilado m\'{a}s de 1.400 publicaciones y sus metadatos, adem\'{a}s de documentaci\'{o}n sobre herramientas y datasets relacionados. Los metadatos son exportables en los formatos BibTeX, RDF, CSV y JSON. Los dumps del wiki con sus historiales est\'{a}n disponibles para descarga.
\end{abstract}
	

\begin{multicols}{2}
\section{Introducci\'{o}n}
\lettrine[nindent=0em,lines=3]{E} l inter\'{e}s de los investigadores por los wikis, en especial Wikipedia, ha ido creciendo en los \'{u}ltimos a\~{n}os. La primera edici\'{o}n de WikiSym, un simposio sobre wikis, se celebr\'{o} en 2005 y desde entonces han aparecido congresos como CLEF/PAN Lab, workshops como WikiAI, SemWiki y MathWikis, conferencias como Wikimania, WikiCon, SMWCon, Wiki Conference India, Wikipedia Academy y Wikipedia CPOV Conference y competiciones como WikiViz. El estudio de los wikis es un campo emergente y prol\'{i}fico.

\section{Trabajos relacionados}
Ha habido varios intentos de recopilar toda la literatura cient\'{i}fica sobre wikis, aunque con poco \'{e}xito. Unas veces el enfoque o la herramienta utilizada eran limitados, otras veces el proyecto era abandonado y al poco tiempo los metadatos bibliogr\'{a}ficos se perd\'{i}an. Se han hecho recopilaciones en p\'{a}ginas personajes y blogs, a trav\'{e}s de revisiones de literatura, haciendo uso de gestores de bibliograf\'{i}a, en p\'{a}ginas de Wikipedia y tambi\'{e}n en servicios como Zotero o CiteULike.

\subsection{P\'{a}ginas personales y blogs}
Existen ejemplos de recopilaciones de literatura en webs personales\footnote{\href{http://www.public.iastate.edu/~CYBERSTACKS/WikiBib.htm}{http://www.public.iastate.edu/~CYBERSTACKS/WikiBib.htm}} y blogs. Un ejemplo bastante completo de este \'{u}ltimo es SWEETpedia,\footnote{\href{http://www.mkbergman.com/sweetpedia/}{http://www.mkbergman.com/sweetpedia/}} que contiene publicaciones sobre wikis y sem\'{a}ntica. Uno de los inconvenientes de este sistema es que requiere mucho esfuerzo para una \'{u}nica persona.

\subsection{Revisiones de literatura}
Se han realizado varias revisiones de literatura hasta el momento. La primera de ellas (Voss, 2005) se hizo en un momento en el que las publicaciones eran escasas, pero ya se intu\'{i}a que estaba en crecimiento. Un a\~{n}o m\'{a}s tarde (Ayers, 2006) vuelve a hacer un repaso a la literatura existente.

No ser\'{i}a hasta 3 a\~{n}os despu\'{e}s cuando (Okoli et al., 2009) presentan una propuesta de protocolo para hacer un mapeo sistem\'{a}tico y ese mismo a\~{n}o hiciese una revisi\'{o}n (Okoli, 2009).

(Nielsen, 2011) hace la mayor revisi\'{o}n de literatura en un documento de m\'{a}s de 50 p\'{a}ginas, en progreso e inacabado, que incluye m\'{a}s de 300 referencias a publicaciones y asegura haber encontrado m\'{a}s de 1.000 publicaciones sobre el tema.

(Martin, 2011)
(Okoli, 2012)
(Jullien, 2012)

Uno de los inconvenientes es que estos documentos quedan rapid\'{i}simamente desactualizados, por el ritmo de publicaci\'{o}n existente.

\subsection{Gestores bibliogr\'{a}ficos}
Se han empleado \href{http://wikindx.inrp.fr/biblio_encyclen/}{http://wikindx.inrp.fr/biblio\_encyclen/}

Wiki Research Bibliography
\href{http://toolserver.org/~voj/bibliography/}{http://toolserver.org/~voj/bibliography/}
\href{http://wikiindex.org/Wiki_Research_Bibliography}{http://wikiindex.org/Wiki\_Research\_Bibliography}
(only selected people to edit)

\subsection{P\'{a}ginas individuales en Wikipedia}
Tambi\'{e}n existen listados de publicaciones y recursos en algunas Wikipedias, como en la versi\'{o}n alemana y la inglesa. El principal inconveniente es que no es posible jugar con los datos dentro del mismo wiki, al estar todo escrito como texto plano, sin enriquecimiento sem\'{a}ntico.

\subsection{Servicios web y redes sociales}
Finalmente servicios web y redes sociales con recopilaciones de literatura sobre wikis. Es el caso de grupos de Zotero, tags de BibSonomy y grupos y tags de CiteULike.

\section{WikiPapers}
WikiPapers fue lanzado en abril de 2011. Haciendo uso de MediaWiki y su extensi\'{o}n sem\'{a}ntica, recopila de manera colaborativa informaci\'{o}n acerca de toda la literatura cient\'{i}fica sobre wikis, as\'{i} como de herramientas y datasets relacionados. No hace falta estar registrado para participar, pero es recomendable.

WikiPapers agrupa todas las ventajas de los sistemas mencionados anteriormente y soluciona sus inconvenientes. Permite hacer listados de publicaciones similares a SWEETpedia: existe uno de revisiones de literatura por poner solo un ejemplo. Funciona como un gestor bibliogr\'{a}fico, al almacenar los metadatos de las publicaciones y permitir hacer b\'{u}squedas, filtrarlos o exportarlos, individualmente o en conjunto. Tambi\'{e}n facilita que grupos de usuarios se comuniquen a trav\'{e}s de las p\'{a}ginas de discusi\'{o}n y compartan informaci\'{o}n sobre publicaciones de su inter\'{e}s, funcionando como una red social. El espacio de discusi\'{o}n debajo de cada p\'{a}gina posibilita a los lectores hacer valoraciones de los art\'{i}culos.

Desde un punto de vista m\'{a}s estad\'{i}stico, es posible generar gr\'{a}ficas a partir de los metadatos disponibles en WikiPapers, aprovechando as\'{i} la capacidad que ofrece la sem\'{a}ntica. Gr\'{a}ficos de barras, circulares o l\'{i}neas temporales est\'{a}n presentes y facilitan la visualizaci\'{o}n y comprensi\'{o}n de la informaci\'{o}n.

Finalmente, el wiki y sus historiales est\'{a}n disponibles tanto para su descarga como dump XML y accesible a trav\'{e}s de la API de MediaWiki. Esto impide que todo el trabajo se pierda, como sucedi\'{o} en alg\'{u}n proyecto del pasado.

\subsection{Publicaciones}
En WikiPapers cada publicaci\'{o}n dispone de una p\'{a}gina en la que se detallan todos sus metadatos (t\'{i}tulo, autores, palabras clave, a\~{n}o, revista o congreso, DOI, idioma, licencia, enlaces al fichero y motores de b\'{u}squeda), el abstract, las referencias que incluye y las citas que recibe, y un espacio de discusi\'{o}n. Los metadatos sirven para hacer b\'{u}squedas y filtrar los contenidos. A octubre de 2012 ya cuenta con m\'{a}s de 1.400 publicaciones, incluyendo art\'{i}culos de revistas y congresos, tesis y libros.

\subsection{Autores}
Para cada autor existe una ficha que incluye su nombre, afiliaci\'{o}n, pa\'{i}s, \'{i}ndice de coautores, p\'{a}gina web, y por supuesto un listado de publicaciones, datasets y herramientas de las que es autor.

\subsection{Herramientas}
Repaso a las herramientas

\subsection{Datasets}
Repaso a los datasets, los wikis como datasets (WIkiTeam)

\subsection{Eventos}
ya est\'{a}n nombrados en el abstract y la intro...

\subsection{Y m\'{a}s...}
Tambi\'{e}n hay informaci\'{o}n sobre conceptos, ejemplos, preguntas abiertas, encuestas, motores wiki, wikifarms.
Los autores pueden presentarse un poco en sus p\'{a}ginas de usuario.
Posibilidad de incrustar diapositivas (por ejemplo SlideShare) y v\'{i}deos (YouTube, Vimeo...).

\subsection{Reutilizaci\'{o}n}
Posibilidades de reutilizar el contenido ….
Todos los metadatos se pueden exportar en los formatos BibTeX, RDF, CSV y JSON.
Comentar que WikiLit ha cogido las plantillas y estructura de WikiPapers para crear su propio review de la literatura, limitado a revistas... ?

\section{Conclusiones y trabajo futuro}
porqu\'{e} hac\'{i}a falta WikiPapers
aglutina todas las ventajas de los anteriores sistemas
lo que se ha hecho, cifras,
lo que queda por hacer y como ayudar
el futuro y m\'{a}s all\'{a}...

\section{Agradecimientos}
...

\section{Referencias}
Voss, J. (2005). Measuring Wikipedia. In the International Conference of the International Society for Scientometrics and Informetrics.

Ayers, P. (2006). Researching Wikipedia -- Current approaches and new directions. Proceedings of the American Society for Information Science and Technology

Okoli, C. and Schabram, K. (2009). Protocol for a systematic literature review of research on the Wikipedia. International Conference on Management of Emergent Digital EcoSystems

Okoli, C. (2009). A Brief Review of Studies of Wikipedia in Peer-Reviewed Journals. Third International Conference on Digital Society
Nielsen, F. A. (2011). Wikipedia research and tools: Review and comments.

Martin, O. S. (2011). A Wikipedia Literature Review.
Okoli, C., Mehdi, M., Mesgari, M., Nielsen, F. A.,  Lanam\"{a}ki, A. (2012). The people's encyclopedia under the gaze of the sages: a systematic review of scholarly research on Wikipedia.

Jullien, N. (2012). What We Know About Wikipedia: A Review of the Literature Analyzing the Project(s).

\section{Licencia}
Este obra est\'{a} bajo licencia \href{http://creativecommons.org/licenses/by-sa/3.0/}{Creative Commons Reconocimiento-CompartirIgual 3.0 Unported}.

\end{multicols}

\end{document}
