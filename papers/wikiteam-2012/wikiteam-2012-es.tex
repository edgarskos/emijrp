\documentclass[11pt,twocolumn]{article}
\setlength{\columnsep}{0.5cm}

\usepackage[utf8]{inputenc}
\usepackage[T1]{fontenc}
\usepackage[spanish]{babel}
\usepackage{hyperref}
\usepackage{graphicx}
\usepackage{natbib}

\title{\vspace{-15mm}%
	\fontsize{24pt}{10pt}\selectfont
	%\textbf{WikiTeam: collaborative preservation of wikis}
	\textbf{WikiTeam: preservación colaborativa de wikis}
	}	
\author{%
	\large
	\textsc{Emilio J. Rodríguez-Posada} \\
	\normalsize	Private \\
	\normalsize	\href{mailto:emijrp@gmail.com}{emijrp@gmail.com}
	\vspace{-5mm}
	}
\date{}


\begin{document}


\twocolumn[
  \begin{@twocolumnfalse}

    \maketitle

\begin{abstract}
  Los internautas tienen cada vez un papel más relevante en la generación del contenido de los sitios web. Existen iniciativas y soluciones para la preservación digital de la web, incluso con largo recorrido como Internet Archive, pero son ineficientes y padecen dificultades a la hora de preservar contenido creado por los usuarios en redes sociales y wikis. En este artículo analizamos los problemas existentes a la hora de preservar wikis, la ausencia de herramientas para realizar esta labor y presentamos y evaluamos la solución que hemos construido, llamada WikiTeam. El software desarrollado en WikiTeam es un conjunto de herramientas para preservar wikis, hasta el momento MediaWiki. Tras ponerlo en práctica hemos logrado extraer los textos, historiales, imágenes y metadatos de más de 4.500 wikis de toda la red. Con la experiencia recabada planeamos no solo ampliar el número de wikis preservados sino expandirnos a otros motores wiki. Todo el contenido recuperado representa un cúmulo enorme de datasets de la wikiesfera, con un incalculable valor tanto histórico como para la investigación de estas comunidades wiki.
  \\
  \\
  \textbf{Palabras clave:} web digital preservation, social web archiving, archiving applications and systems

\end{abstract}

  \end{@twocolumnfalse}
  ]

\section{Introducción}

\section{Preservación de wikis}

%no compila esta línea : / \section{WikiTeam preservando la wikiesfera}

\section{Las 5 estrellas de la preservación de wikis}

\section{Conclusiones y trabajo futuro}

\bibliographystyle{wink}        
\bibliography{wikiteam-2012}

\section{Licencia}
Esta obra está bajo licencia \href{http://creativecommons.org/licenses/by-sa/3.0/}{Creative Commons Reconocimiento-CompartirIgual 3.0 Unported}.

\end{document}
